\section{\label{Deployment}Deployment}
This section describes the possible deployment strategies. 

\begin{note}
this section is a subject to change
\end{note} 

\subsection{Continuous Integration Server}
The \emph{Continuous Integration Server} does a publish in a directory
structure. This directory structure is transformed by a \emph{fetch} script into the designated
directory structure.

It's not recommended to use the \emph{Continuous Integration Servers} structure
directly. The next section will discuss the fetching process and the designated
directory structure.

\subsection{Fetching and Destination Directory Structure}

The fetching script is responsible for
\begin{itemize}
  \item Creating the directory structure
  \item Fetching all Assemblies and putting them in the right destination
  directory
  \item Copying all app configuration files to the right directories 
\end{itemize} 

The directory structure on the deployment server should look like this:

  	\begin{itemize}
  	  \item AppConfigs
  	  \item { bin
  	  	\begin{itemize}
  	  	  \item Bootstrapper
  	  	  \item { Client
  	  	  	\begin{itemize}
  	  	  	  \item { Client
  	  	  	  	\begin{itemize}
  	  	  	  	  \item App.ZBox
  	  	  	  	  \item Core
  	  	  	  	  \item Core.Generated
  	  	  	  	  \item WPF
  	  	  	  	  \item WinForms
  	  	  	  	 \end{itemize}
  	  	  	  }
  	  	  	  \item { Common
  	  	  	  	\begin{itemize}
  	  	  	  	  \item App.ZBox
  	  	  	  	  \item Core
  	  	  	  	  \item Core.Generated
  	  	  	  	 \end{itemize}
  	  	  	  }
  	  	  	  \item Kistl.Client.WPF.exe
  	  	  	  \item Kistl.Client.WPF.exe.config
  	  	  	  \item Kistl.Client.Forms.exe
  	  	  	  \item Kistl.Client.Forms.exe.config
  	  	  	\end{itemize}  
  	  	  }
  	  	  \item {Server
  	  	  	\begin{itemize}
  	  	  	  \item { Common
  	  	  	  	\begin{itemize}
  	  	  	  	  \item App.ZBox
  	  	  	  	  \item Core
  	  	  	  	  \item Core.Generated
  	  	  	  	 \end{itemize}
  	  	  	  }
  	  	  	  \item { Server
  	  	  	  	\begin{itemize}
  	  	  	  	  \item App.ZBox
  	  	  	  	  \item Core
  	  	  	  	  \item Core.Generated
  	  	  	  	  \item EF
  	  	  	  	  \item EF.Generated
  	  	  	  	  \item NH
  	  	  	  	  \item NH.Generated
  	  	  	  	 \end{itemize}
  	  	  	  }
  	  	  	  \item Kistl.Server.Service.exe
  	  	  	  \item Kistl.Server.Service.exe.config
  	  	  	\end{itemize}  
  		}
  	  	\end{itemize}
  	  }
  	  \item Configs
  	  \item DocumentStore
  	  \item { inetpub
  	  	\begin{itemize} 
  	  	  \item App\_Data
  	  	  \item App\_GlobalResources
  	  	  \item App\_Themes
  	  	  \item bin
  	  	  \item Bootstrapper
  	  	  \item { Common
  	  	  	\begin{itemize}
  	  	  	  \item App.ZBox
  	  	  	  \item Core
  	  	  	  \item Core.Generated
  	  	  	 \end{itemize}
  	  	  }
  	  	  \item { Server
  	  	  	\begin{itemize}
  	  	  	  \item App.ZBox
  	  	  	  \item Core
  	  	  	  \item Core.Generated
  	  	  	  \item EF
  	  	  	  \item EF.Generated
  	  	  	  \item NH
  	  	  	  \item NH.Generated
  	  	  	 \end{itemize}
  	  	  }
  	  	  \end{itemize} 
  	  }
  	  \item logs
  	  \item Packages
  	  \item deploy.ps1
  	  \item fetch.ps1
  	\end{itemize} 

\subsubsection{Root Directory}
\begin{descriptionBorder}
\item[deploy.ps1] The deployment script, responsible for upgrading the servers
database

\item[fetch.ps1] The fetch script, responsible for creating the directory
structure and fetching all assemblies and configuration templates
\end{descriptionBorder}

\subsubsection{AppConfigs}
This directory contains all app configuration files needed by the executing
assemblies. E.g. in the \emph{Kistl.Server.Service.exe.config} all WCF Service
stuff can be configured. In \emph{Kistl.Client.WPF.exe.config} all WCF Proxy
stuff can be configured.

The fetch script will copy those configuration files into the desired
directories, right beside their executalables.

So this is the place where all app specific configuration has to be defined.

Besides WCF stuff, those configuration can be also found in those files:

\begin{itemize}
  \item log4net
  \item WCF
  \item Assembly Bindings
  \item Database provider registration 
\end{itemize} 

\begin{note}
The web.config is not located here, but this may change in the
future
\end{note}

\subsubsection{Configs}
This directory contains all ZBox configuration files. They are located by the
executalable by probing 
\begin{itemize}
  \item the given command line parameter
  \item then the \emph{zenv} environment variable plus executalable name
  \item then each directory up to the \emph{Configs} directory, still with the
  executalable name
  \item at the end by looking for \emph{DefaultConfig.xml} in the configs
  directory
\end{itemize}

\subsubsection{bin}
The bin directory contains all assemblies used by ZBox, divided by theire use
(Bootstrapper, Client or Server). The \emph{Client} and \emph{Server}
directories contains sub directories to split Client/Server and Common parts.
Those directories have for each Application (=ZBox Module) and Modules (not ZBox
Modules!) a individual sub directory.

The naming of Application (=ZBox Module) should be:

\begin{center}
App.\textless ModuleName\textgreater  
\end{center}

\emph{Core} contains all core assemblies like Kistl.API and all other
assemblies that are referenced by default (log4net e.g.).

\emph{Core.Generated} contains all genereated code. Code Generation is done by
the \emph{Continuous Integration Server} so those assemblies will be copied by
the fetch script.

\begin{note}
This is a topic of discussion
\end{note}

\emph{Bootstrapper} contains only one executalable. The Boostrapper itself which
is downloading the whole client application with its configuration from an HTTP
Server via REST.

\subsection{Deployment on a Linux Server}

On Linux the following packages are needed:

\begin{itemize}
  \item Mono 2.10 - master
  \item Apache 2
  \item mod\_mono
  \item many other\ldots
 \end{itemize}

\subsubsection{Apache configuration}

\begin{itemize}
\item Install Apache
\item install mod\_mono
\item turn off KeepAlive - the .NET HTTP Client won't talk to Apache correctly 
\item enable/configure your security model
\item ensure that the
\emph{TrustedBasicAuthenticationModule} is enabled.
\end{itemize}

The Apache Server is responsible for authentication. The ZToolbox server trusts
Apache and uses the passed identity (through the HTTP Header).
 

\subsection{Deployment on a Windows Server}
