\chapter{Programming}

This chapter describes the various ways and pieces the Kistl system is
programmed and customized.

\section{Objects}

\section{Modules}

\section{Enhancing Kistl's inner workings}

\subsection{Database Providers}

\section{Graphical User Interface}

Like other subsystems, the GUI core is designed to be platform
independent. Therefore only the "outermost" shell contains toolkit
specific code.

\subsection{Architecture}

The GUI is modeled after the Model-View-ViewModel architecture. The
\emph{Model} represents the underlying data structures and business
logic. It is provided by the generated classes from the actual
datamodel. \emph{View Models} provide display specific functionality
like formatting, transient state holding and implementing the user's
possible actions. They always inherit from
\texttt{Kistl.Client.Presentables.PresentableModel}. Common
implementations reside in the \texttt{Kistl.Client.Presentables}
namespace. Finally, \emph{Views} (editors and displays) are the actual
components taking care of showing content to the user and converting the
users keypresses and clicks into calls on the view models interface.
Views are toolkit\footnote{Toolkits are GUI libraries like GTK\# or
Windows Forms.} specific and reside in the toolkit's respective
assembly.

